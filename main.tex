\documentclass[11.5pt]{article}

\usepackage{curriculum}

\newcommand\en[1]{}
\newcommand\fr[1]{#1}

\begin{document}

    \begin{minipage}[t]{\textwidth}
        \begin{minipage}{\cvPhotoWidth}
            \includegraphics[width=\textwidth]{src/profile.png}
        \end{minipage}
        \hspace{\cvHspacePhotoTitle}
            \begin{minipage}[t]{\textwidth-\cvPhotoWidth-\cvHspacePhotoTitle}
            \begin{minipage}{\cvNameWidth}
            \begin{Huge}
                \vspace{.5cm}
                \textbf{Henri Lefebvre}
            \end{Huge}
            23 \en{years old}\fr{ans}
            \center{
                \en{ Applies for "prix master ROADEF" }
                \fr{ Candidat au prix master ROADEF }
            }
            \end{minipage}
            \begin{minipage}{\cvAddressWidth}
                \begin{flushright}
                    33 Rue des Cronquelets, \\
                    Saint-Aubin, 62170 (FR)\\
                    \underline{henri.lefebvre@yahoo.com}\\
                    (+33) 07 78 34 29 49\\
                    \fr{Permis B}
                    \en{Driving licence}
                \end{flushright}
            \end{minipage}
        \end{minipage}
    \end{minipage}
    \begin{minipage}[t]{\textwidth}
        \vspace{\cvVspacePhotoTitle}
        \begin{minipage}[t]{\cvLeftWidth}
            \cvSection{
                \fr{Expériences professionnelles}
                \en{Work experience}
            }
            \cvItem{\fr{Février 2019 - Juillet 2019 (6 mois)}\en{February 2019 - July 2019 (6 months)}}{\fr{Stagiaire - Optimisation robuste}\en{Research intern in robust linear optimization}}{INRIA - RealOpt}{Bordeaux (France)}{
                \begin{itemize}
                    \fr{
                        \item $1|r_j|\sum w_jU_j$ robuste avec recours entiers
                        \item Problèmes de types $\min_{x\in\mathcal X}\max_{\xi\in\Xi}\min_{y\in\mathcal Y(x)}f(x,\xi,y)$
                        \item Branch-and-price, reformulation Dantzig-Wolfe
                        \item Analyse polyhédrale
                        \item Comparaison avec la $K$-adaptabilité
                    }
                    \en{
                        \item Robust $1|r_j|\sum w_jU_j$ with integer recourses
                        \item $\min_{x\in\mathcal X}\max_{\xi\in\Xi}\min_{y\in\mathcal Y(x)}f(x,\xi,y)$ problems
                        \item Branch-and-price, Dantzig-Wolfe reformulation
                        \item Polyhedral analysis
                        \item $K$-adaptability approach comparison
                    }
                \end{itemize}
                \textit{C++, IBM Cplex}
            }
            \cvItem{\fr{Septembre 2018 - Aout 2018 (12 mois)}\en{September 2018 - August 2018 (12 months)}}{\fr{Développeur backend sous Amazon Web Services}\en{Backend developper on Amazon Web Services}}{Wide Asset Management}{Paris (France)}{
                \begin{itemize}
                    \fr{
                        \item Création du backend de l'application mobile
                        \item Implémentation d'une solution de e-signature
                        \item Développement du moteur d'investissement
                        \item Gestion d'événements asynchrones/parrallèles
                    }
                    \en{
                        \item Mobile application backend development
                        \item E-signature solution implementation
                        \item Investment engine development
                        \item Parallel/asynchronous events management
                    }
                \end{itemize}
                \textit{Finance, AWS, NoSQL, DynamoDB, node.js}
            }
            % \cvItem{Février 2018 (1 mois)}{Stagiaire - assistant contrôle qualité}{Valeo}{Étaples}{}
            \cvSection{\fr{Projets et réalisations}\en{Projects and achievements}}
            \cvItem{\fr{Décembre 2018 - Mars 2019 (4 mois)}\en{December 2018 - March 2019 (4 months)}}{\fr{Heuristique pour Tournée de Vehicules Electriques}\en{Electrical Vehicle Routing Problem heuristic}}{Università degli studi di Genova}{\fr{Gènes (Italie)}\en{Genova (Italy)}}{
                \begin{itemize}
                    \fr{
                        \item Problème de tournées de véhicules avec stations de rechargements saturées et fenêtres de temps pour les livraisons
                    }
                    \en{
                        \item Vehicle Routing Problem with capacitated recharging stations and time windows for deliveries
                    }
                    \item \textit{Simulated Annealing} (SA)
                    \item \textit{Variable Neighbourhood Search} (VNS)
                \end{itemize}
                \textit{C++, E-VRP-TW, 2-OPT moves}
            }
            \cvItem{\fr{Février 2018 - Juin 2018 (5 mois)}\en{February 2018 - June 2018 (5 months)}}{\fr{Planificateur d'itinéraire multimodal}\en{Multimodal Itinerary Planning}}{\fr{Laboratoire }Heudiasyc\en{ laboratory}}{Compiègne (France)}{
                \begin{itemize}
                    \fr{
                        \item Algorithme de Dijkstra avec dépendence temporelle
                        \item Exploitation de données \textit{Google Transit Format Specification} (GTFS) pour la génération du graphe
                    }
                    \en{
                        \item Dijkstra's algorithm with time dependencies
                        \item Real \textit{Google Transit Format Specification} (GTFS) instances for graph generation (Paris, Rome, ...)
                    }
                \end{itemize}
                \textit{Python, SQLite, time expanded/dependent model}
            }
            \cvItem{\fr{Avril 2017 - Juin 2017 (3 mois)}\en{April 2017 - June 2017 (3 months)}}{\fr{Intelligence artificielle pour jeu de plateau}\en{Board game artifical intelligence}}{\fr{Université de Technologie}\en{University of Technology}}{Compiègne (France)}{
                \begin{itemize}
                    \fr{
                        \item Joueur autonome pour le jeu Arimaa
                        \item Programmation logique pour la création de l'IA
                    }
                    \en{
                        \item Autonomous player for Arimaa board game
                        \item Logic programming for adversary's strategy
                    }
                \end{itemize}
                \textit{Prolog, \fr{intelligence artificielle symbolique}\en{symbolic artifical intelligence}}
            }
        \end{minipage}
        \hspace{\cvHspaceLeftRight}
        \begin{minipage}[t]{\cvRightWidth}
            \cvSection{\fr{Diplômes}\en{Education}}
            \cvItem{2019 - 2022 (3 \fr{ans}\en{years})}{\fr{Doctorat en optimisation}\en{Ph.D Student in operations research}}{\fr{Università di Bologna}\en{Università di Bologna}}{Bologne (\fr{Italie}\en{Italy})}{
                \begin{itemize}
                    \fr{
                        \item Décomposition de Benders pour problèmes convexes non linéaires mixtes
                        \item Tuteurs : Enrico Malaguti, Michele Monaci
                    }
                    \en{
                        \item "Benders decomposition for convex nonlinear mixed integer problems"
                        \item Supervisors : Enrico Malaguti, Michele Monaci
                    }
                \end{itemize}
            }
            \cvItem{2014 - 2019 (5 \fr{ans}\en{years})}{\fr{Diplôme d'Ingénieur - Génie Informatique}\en{Engineer's degree in Computer Science}}{\fr{Université de Technologie}\en{University of Technology}}{Compiègne (France)}{
                \begin{itemize}
                    \fr{
                        \item Filière \textit{Aide à la Décision en logistique}
                        \item Mineur \textit{Philosophie, Technologie et cognition}
                        \item 1 Semestre à l'\textcolor{red!60!black}{Université de Shanghai} (Chine)
                    }
                    \en{
                        \item Major : \textit{Decision support applied to logistics}
                        \item Minor : \textit{Philosophy, Technology and Cognition}
                        \item Study semester in \textcolor{red!60!black}{Shanghai University} (China)
                    }
                \end{itemize}
            }
            \cvItem{2018 - 2019 (12 \fr{mois}\en{months})}{\fr{Master -} \textit{Laurea Magistrale en Informatica}}{Università degli studi di Genova}{\fr{Italie}\en{Italy}}{
                \begin{itemize}
                    \fr{
                        \item Filière \textit{Logistics and production}
                        \item Double diplôme
                    }
                    \en{
                        \item Major : \textit{Logistics and production}
                    }
                \end{itemize}
            }
            \cvItem{2018 - 2019 (12 \fr{mois}\en{months})}{\fr{Master - Ingénieurie des Systèmes Complexes}\en{Master degree - Complex systems Engineering}}{\fr{Université de Technologie}\en{University of Technology}}{Compiègne (France)}{
                \begin{itemize}
                    \fr{
                        \item Filière \textit{Apprentisage et Optimisation des Systèmes}
                    }
                    \en{
                        \item Major : \textit{Optimization and learning of systems}
                    }
                \end{itemize}
            }
            \cvSection{\fr{Compétences}\en{Skills}}
            \textbf{Sciences}
                \vspace{10pt}
                \begin{itemize}
                    \fr{
                        \item Optimisation {\small (\emph{LP}, \emph{MILP})}
                        \item Méthodes heuristiques et metaheuristiques
                        \item Méthodes de décompositions
                        \item Modélisation mathématique
                        \item Apprentisage automatique et statistiques
                        \item Analyse et calcul numérique
                    }
                    \en{
                        \item Linear and non linear optimization {\small (\emph{LP}, \emph{MILP}, \emph{KKT})}
                        \item Optimization under uncertainty (robust, stochastic)
                        \item Decomposition methods (Dantzig-Wolfe, Benders)
                        \item Heuristics and metaheuristic methods
                        \item Mathematical modeling
                        \item Exact approaches
                        \item Machine learning and statistics
                        \item Numerical analysis
                    }
                \end{itemize}
                \vspace{10pt}
            \textbf{Programmation}
            \begin{multicols}{2}
                \begin{itemize}
                    \item C++
                    \item Python 3
                    \item IBM Cplex
                    \item Lingo
                    \item\texttt{GNU R}
                    \item SciLab
                    \item\LaTeX
                    \item Linux
                    \item Git
                    \item Amazon Web Services
                \end{itemize}
            \end{multicols}
            \textbf{Langues}
            \begin{multicols}{2}
                \begin{itemize}
                    \item \fr{Anglais}\en{English} {\footnotesize(TOEIC 965)}
                    \item \fr{Français}\en{French}
                    \item \fr{Italien}\en{Italian} {\footnotesize(basic)}
                    \item \fr{Espagnol}\en{Spanish} {\footnotesize(basic)}
                \end{itemize}
            \end{multicols}
            % \textbf{\fr{Intérêts}\en{Interests}}
            % \begin{multicols}{2}
            %     \begin{itemize}
            %         \fr{
            %             \item Environnement
            %             \item Sciences cognitives
            %             \item Vulgarisation
            %             \item Poésie française
            %         }
            %         \en{
            %             \item Environment
            %             \item Cognitive sciences
            %             \item Vulgarization
            %             \item French poetry
            %         }
            %     \end{itemize}
            % \end{multicols}
        \end{minipage}
    \end{minipage}

\end{document}
